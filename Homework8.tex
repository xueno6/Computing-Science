\documentclass[twoside,a4paper]{article}
\usepackage{geometry}
\geometry{margin=1.5cm, vmargin={0pt,1cm}}
\setlength{\topmargin}{-1cm}
\setlength{\paperheight}{29.7cm}
\setlength{\textheight}{25.3cm}

% useful packages.
\usepackage{amsfonts}
\usepackage{amsmath}
\usepackage{amssymb}
\usepackage{amsthm}
\usepackage{enumerate}
\usepackage{graphicx}
\usepackage{multicol}
\usepackage{fancyhdr}
\usepackage{layout}
\usepackage{mathrsfs}
% some common command
\newcommand{\dif}{\mathrm{d}}
\newcommand{\avg}[1]{\left\langle #1 \right\rangle}
\newcommand{\difFrac}[2]{\frac{\dif #1}{\dif #2}}
\newcommand{\pdfFrac}[2]{\frac{\partial #1}{\partial #2}}
\newcommand{\OFL}{\mathrm{OFL}}
\newcommand{\UFL}{\mathrm{UFL}}
\newcommand{\fl}{\mathrm{fl}}
\newcommand{\op}{\odot}
\newcommand{\Eabs}{E_{\mathrm{abs}}}
\newcommand{\Erel}{E_{\mathrm{rel}}}


\begin{document}


\pagestyle{fancy}
\fancyhead{}
\lhead{Xue Sen (3160104429)}
\chead{Numerical Approximation homework \#8}
\rhead{Date}

\section*{I. Briefly repeat the problem}
Firstly, a vector space should be proved.\\
Since $$ L^{2} _\rho =\{ f(x): \rho (x) |f(x)|^2 \in L [a,b]\}$$
For $u(x)$, $v(x)$ and $w(x)$ :\\

$$u+v=\rho |u+v|^2=v+u$$
$$(u+v)+w=\rho |u+v+w|^2=u+(v+w)$$
$$(ab)u=\rho |abu|^2=a(bu)$$
$$Supposing \ f=0, u+f=\rho |u|^2=u$$
$$Supposing \ v=-u, v+u= \rho |0|^2=\textbf{0}$$
$$ 1u=\rho |u|^2=u$$
$$ for \forall a,b, \ and \ u,v  \quad a(u+v)=au+av \quad (a+b)u=au+bu$$
So It is a vector space.\\
\\
Secondly, it is an inner product space\\
For $\forall v,u,w\\$
(i) $\forall v\in \mathcal{L},\quad  \left< v,v \right>=  \int_{a}^b \rho (t) v^{2}(t) dt $\\
Since $\rho(x) \ge 0$, $ \left< v,v \right> \ge 0$\\
(ii) Since $\rho \neq 0$, $\int_{a}^{b} \rho (t) v^2(t) dt=0$ $\rightarrow v =0$\\
So $\left< v,v \right>=0$ only when v=0.\\
(iii)$\left< u+v,w \right>= \int_{a}^b \rho (t) (u(t)+v(t))w(t) dt=\int_{a}^b \rho (t) u(t)w(t)+v(t)w(t) dt=\left< u,w \right>+\left< v,w \right>\\$
(iv) $\left< av,w \right>= \int_{a}^b \rho (t) av(t)w(t) dt=a \int_{a}^b \rho (t) v(t)w(t) dt=a\left< av,w \right>$
(v) $\left< v,w \right>=\int_{a}^b \rho (t) v(t)w(t) dt\\$
Because $v(t)w(t)=\overline {v(t)w(t)}$
$\left< v,w \right>=\left< \overline {v,w} \right>\\$



\section*{II. In this question, the Chebyshev polynomials are considered}
\subsection*{II-a}
Consider that $T_n=cos(narccos(x))\\$
a=-1, b=1 and $\rho =  \frac{1}{\sqrt{1-x^2}}\\$
(i)So that $ \int_{-1}^1 \rho (x) T_n(x)^2 dx=\int_{-1}^1 \displaystyle\frac{cos(narccosx)^2}{\sqrt{1-x^2}} dx$\\
let $x=cos \theta$\\
 $ \int_{\pi}^2\pi \displaystyle\frac{cos(n \theta)}{-sin \theta} dcos\theta=\int_{\pi}^2\pi cos(n\theta)^2 d\theta =\pi/2 $\\
 \\
(ii)And also that \\
 $\int_{-1}^{1} \rho T_n T_m dx=\int_{-1}^{1} \displaystyle \frac{cos(n arccos(x)) cos(m arccos(x))}{\sqrt{1-x^2}} dx
 $\\
Let $x=cos \theta$\\
 $ \int_{-1}^{1} \rho T_n T_m dx=\int_{\pi}^{2\pi} cos(n\theta) cos(m \theta) d \theta =0 \quad (m \neq n)$\\

\subsection*{II-b}
With a normal order:
$u_1=1, \quad v_1= 1,  \quad ||v_1||^2  =\int_{-1}^{1} \frac{1}{\sqrt{1-x^2}} dx =\int_{\pi}^{2\pi} 1 d \theta =\pi, and \quad u^{*} _1 =1/\sqrt{\pi} $\\
$u_2=x, \quad v_2=u_2-\left< u_2,u^{*}_1 \right> u^{*}_1=x, \quad ||v_2||^2=\int_{-1}^{1} \displaystyle {x^2}{\sqrt{1-x^2} } dx =\pi/2, and \quad u^{*}_2=\sqrt{2/\pi} x$\\
$u_3=2x^2-1, \quad v_3=u_3-\left< u_3,u^{*}_1 \right> u^{*}_1-\left< u_3,u^{*}_2 \right> u^{*}_2=u_3, \quad ||v_3||^2=\int_{-1}^{1} \displaystyle \frac{(2x^2-1)^2}{\sqrt{1-x^2}} dx =\pi/2, \quad u^{*}_3 =\sqrt{2/\pi}(2x^2-1)$\\
\section*{III. In this question Fourier expansion and normal equations are used to get Least-square approximation}
\subsection*{III-a}
As in question II\\
$u^{*}_1=1/\sqrt{\pi}, \quad u^{*}_2=x\sqrt{2/\pi}, \quad u_3 ^{*}=(2x^2-1)\sqrt{2/\pi}\\$
$b_0=2/\sqrt{\pi}, \quad b_1=int_{-1}^{1} x\sqrt{2/\pi} dx=0, \quad b_2=int_{-1}^{1} (2x^2-1)\sqrt {2/\pi} dx =-2/3 \sqrt{2/\pi}\\$
$\phi_2=2/\pi -\frac{4}{3*\pi}(2x^2-1)$\\
\subsection*{III-b}
Starting with a list $(1,x,x^2)$
Since \\G=$
\begin{bmatrix}
\left< 1,1 \right> & \left< 1,x \right> &\left< 1,x^2 \right>\\
\left< x,1 \right> & \left< x,x \right> & \left< x,x^2 \right> \\
\left< x^2,1 \right> & \left<x^2,x \right> & \left<x^2,x^2 \right>
\end{bmatrix}
=
\begin{bmatrix}
\pi & 0&\pi/2\\
0 & \pi/2 & 0 \\
\pi/2 & 0 & 3/8 \pi
\end{bmatrix}$\\
C=$
\begin{bmatrix}
\left< \sqrt{1-x^2},1 \right> \\
\left< \sqrt{1-x^2},x \right> \\
\left< \sqrt{1-x^2},x^2 \right>
\end{bmatrix}
=
\begin{bmatrix}
2\\
0\\
2/3
\end{bmatrix}
$\\
\\
Because \textbf{Ga=C}\\
Solve it by linear equations and we have:\\
a=$
\begin{bmatrix}
10/3 \pi\\
0\\
-8/3 \pi
\end{bmatrix}
$

\section*{IV. This question is about Example 6.19 and it is required to get independent list and best approximation.}
\subsection*{IV-a}
As it is in the notes:\\
$ u_1=1, \quad v_1=u_1, \quad ||v_1||^2=12, \quad and \quad u^{*}_1=1/(2 \sqrt{3})\\$
$ u_2=x, \quad v_2=u_2- \left<u_2,u^{*}_1 \right> u^{*}_1=x-13/2, \quad ||v_2||^2=143, \quad and \quad u^{*}_2=\frac {1}{\sqrt{143}}(x-13/2)$\\
$u_3=x^2, \quad v_3=u_3-\left<u_3,u^{*}_1 \right> u^{*}_1-\left<u_3,u^{*}_2 \right> u^{*}_2=x^2-13x+30.33, \quad and ||v_3||^2=1334.6668, \quad u^{*}_3=(x^2-13x+30.33)/ \sqrt{1334.6668} \\$
\subsection*{IV-b}
$w=\sum_{i=1}^{3}\left<y,u^{*}_i \right>u^{*}_i=138.5+4.1189(x-6.5)+9.0378(x^2-13x+30.25)= 9.0378x^2-113.3725x+ 385.1206$\\
The above result is very close to the example's solution.\\
\\
As Theorem 6.14 suggests, $||y-\hat{\phi}|| \le ||y-\sum_{i=0}^{2}b_i x^i|| \forall b_i \in R$\\
And it can be proved that $\hat{\phi}$ is the same as that of the example on the table of sales record in the notes. Because as Theorem 6.23 suggest, the vector of best approximation coefficients \textbf{a} is uniquely determined by \textbf{Ga=c}, as Corollary 6.19 yields $$ \left<w,u_j \right>=\sum_{k=1}^{n} a_k \left<u_k,u_j \right>$$
So they are the same thing.
\subsection*{IV-c}
If the $y_i$ are changed, in these calculations above, \textbf{G} and $u^{*}_i$ can be reused and \textbf{c} must be recalculated.\\
By comparing the two method of calculation, the advantage of orthonormal polynomial is obviously, those $u^{*}_i$ will not change if $y_i$ is changed, we only need to reuse $\sum_{i=1}^{3}\left<y,u^{*}_i \right>u^{*}_i$, but in normal equations, we need to get a new \textbf{c}, and solve a linear equation again, which will waste a lot of time.



\end{document}

%%% Local Variables:
%%% mode: latex
%%% TeX-master: t
%%% End:
